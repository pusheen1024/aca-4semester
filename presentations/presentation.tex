\documentclass{beamer}
\usetheme{metropolis}
\usecolortheme{lily}

\usepackage {preamble}

\begin{document}

\author{}
\title{Сортировка подсчётом. Различные реализации}
\date{}
\maketitle

\begin{frame}
\frametitle{Общие сведения. Асимптотика}
Сортировка подсчётом "--- алгоритм сортировки целых чисел в некотором заданном диапазоне. Асимптотика сортировки подсчётом составляет $O(n + k)$, где $n$ "--- количество элементов, а $k$ "--- размер диапазона. В отличие от большинства алгоритмов сортировки, сортировка подсчётом не основана на сравнениях элементов, что и позволяет достичь линейной асимптотики (при использовании сравнений невозможно достичь асимптотики, лучшей чем $O(n \cdot log n)$).
\end{frame}

\begin{frame}
\frametitle{Алгоритм}
\begin{enumerate}
	\item Найдём границы диапазона чисел, содержащихся в массиве $a$ (минимальный и максимальный элементы).
	\item Создадим массив частот размера $max(a) - min(a) + 1$.
	\item Обработаем все числа массива $a$, для каждого числа сделаем прибавление в массиве частот по индексу, соответствующему значению элемента, уменьшенному на минимум (так минимум будет сопоставлен $0$, это позволяет обрабатывать в том числе отрицательные числа).
	\item Заведём счётчик, указывающий на текущий индекс элемента.
	\item Пройдёмся по всем числам от $0$ до $max(a) - min(a)$. Во внутреннем цикле, выполняющемся такое количество раз, которое записано в массиве частот для текущего числа, присвоим текущему элементу $a$ значение данного числа, увеличенного на минимум (так мы вернёмся к исходным значениям массива). Инкрементируем текущий индекс для массива $a$.
\end{enumerate}
\end{frame}

\begin{frame}
\frametitle{Анализ асимптотики всех шагов алгоритма}
\begin{enumerate}
	\item Поиск максимума и минимума в исходном массиве "--- $O(n)$
	\item Создание массива частот "--- $O(k)$
	\item Заполнение массива частот "--- $O(n)$
	\item Основной внешний цикл по всем числам диапазона "--- $O(k)$
	\item Внутренний цикл "--- $O(n)$ суммарно для всех итераций внешнего цикла.
\end{enumerate}
Итого асимптотика алгоритма составляет $O(n + k)$.
\end{frame}

\begin{frame}
\frametitle{Простейшая реализация (только для неотрицательных чисел)}
Эта реализация сортировки подсчётом применяется только для неотрицательных чисел, её асимптотика составляет $O(n + max(a))$.

\
\inputminted{cpp}{./counting_sort_positive.cpp}
\end{frame}

\begin{frame}
\frametitle{Реализация для любых целых чисел, в т.ч. отрицательных}
Эта реализация сортировки подсчётом может применяться для массива, состоящего как из положительных, так и из отрицательных чисел, в ней все числа нормализуются так, что минимум сопоставляется $0$, а остальные числа уменьшаются на значение минимума. Помимо работы с отрицательными числами такая сортировка также позволяет обрабатывать даже очень большие числа, если сам их диапазон невелик. Асимптотика такой реализации составляет $O(n + k)$, где $k = max(a) - min(a)$.
\inputminted{cpp}{./counting_sort_universal.cpp}
\end{frame}

\begin{frame}
\frametitle{Сортировка сложных данных}
Сортировку подсчётом также можно использовать для сортировки структур. Она является устойчивой, т.е. структуры, имеющие одинаковый ключ (по которому и происходит сортировка), сохраняют свой относительный порядок: в результирующем массиве они следуют в том же порядке, что и в исходном. Рассмотрим для примера сортировку пар по первому значению в паре.
Для простоты продемонстрируем его работу при условии, что первое число в каждой паре неотрицательное, на пары, содержащие  алгоритм адаптируется способом, подобным способу для чисел.
Возможно реализовать этот алгоритм двумя способами. 
\textbf{Первый способ}
\begin{enumerate}
\item Создадим массив, элементами которого изначально являются пустые вектора. Размер массива совпадает с размером диапазона
\end{enumerate}
\end{frame}

\end{document}
