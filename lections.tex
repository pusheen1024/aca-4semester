\documentclass{article}

\usepackage {preamble}

\begin{document}

\author{}
\title{Структуры данных и алгоритмы. Анализ сложности алгоритмов}
\date{4 семестр}
\maketitle

Основные свойства алгоритмов:

\begin{enumerate}
\item Дискретность
\item Детерминированность (определённость)
\item Понятность
\item Завершаемость (результативность/конечность)
\item Массовость (универсальность)
\item Однозначность результата
\end{enumerate}

Примеры неразрешимых задач:

\begin{itemize}
\item проблема остановки (зациклится ли данная программа на этом входе или завершится с некоторым результатом);
\item проблема распределения 9 в числе $\pi$ (есть ли в числе $\pi$ цепочка из n подряд идущих девяток);
\item проблема нахождения совершенных чисел;
\item десятая проблема Гильберта (общее решение диофантовых уравнений вида $P(x_1, x_2, \dots, x_n) = 0$);
\item теорема Гёделя о неполноте формальной арифметики;
\item проблема единичной матрицы.
\end{itemize}

Классы алгоритмических задач: вычисление значений функций, распознавание принадлежности объекта заданному множеству.

Математические модели алгоритмов: машина Поста, рекурсивные функции Чёрча, машина Тьюринга.

Машина Тьюринга, способы задания:
\begin{enumerate}
\item Список команд.
\item Таблица переходов.
\item Граф переходов.
\end{enumerate}

Конфигурация МТ "--- совокупность текущего внутреннего состояния, состояния ленты и положения указателя на ленте.

Набор простейших функций:
\begin{enumerate}
\item Нуль"=функция.
\item Функция непосредственного следования.
\item Функция выбора (тождества).
\end{enumerate}

Операции (сохраняют всюду определённость и вычислимость):
\begin{enumerate}
\item Суперпозиция.
\item Примитивная рекурсия.
\item Минимизация.
\end{enumerate}

Функция называется примитивно"=рекурсивной, если она является элементарной или может быть получена из элементарных функций с помощью конечного числа применений операторов тождества, суперпозиции и примитивной рекурсии.
$x * y$, $x + y$ "--- вычислимые функции.
\end{document}
