\documentclass[otchet]{SCWorks}
% Тип обучения (одно из значений):
%    bachelor   - бакалавриат (по умолчанию)
%    spec       - специальность
%    master     - магистратура
% Форма обучения (одно из значений):
%    och        - очное (по умолчанию)
%    zaoch      - заочное
% Тип работы (одно из значений):
%    coursework - курсовая работа (по умолчанию)
%    referat    - реферат
%  * otchet     - универсальный отчет
%  * nirjournal - журнал НИР
%  * digital    - итоговая работа для цифровой кафедры
%    diploma    - дипломная работа
%    pract      - отчет о научно-исследовательской работе
%    autoref    - автореферат выпускной работы
%    assignment - задание на выпускную квалификационную работу
%    review     - отзыв руководителя
%    critique   - рецензия на выпускную работу

% * Добавлены вручную. За вопросами к @mchernigin
\usepackage{preamble}

\begin{document}

% Кафедра (в родительном падеже)
\chair{математической кибернетики и компьютерных наук}

% Тема работы
\title{Сравнение времени работы различных сортировок}

% Курс
\course{2}

% Группа
\group{251}

% Факультет (в родительном падеже) (по умолчанию "факультета КНиИТ")
\department{факультета компьютерных наук и информационных технологий}

% Специальность/направление код - наименование
% \napravlenie{02.03.02 "--- Фундаментальная информатика и информационные технологии}
% \napravlenie{02.03.01 "--- Математическое обеспечение и администрирование информационных систем}
% \napravlenie{09.03.01 "--- Информатика и вычислительная техника}
\napravlenie{09.03.04 "--- Программная инженерия}
% \napravlenie{10.05.01 "--- Компьютерная безопасность}

% Для студентки. Для работы студента следующая команда не нужна.
\studenttitle{студентки}

% Фамилия, имя, отчество в родительном падеже
\author{Потапкиной Маргариты Андреевны}

% Заведующий кафедрой 
\chtitle{доцент, к.\,ф.-м.\,н.}
\chname{С.\,В.\,Миронов}

% Руководитель ДПП ПП для цифровой кафедры (перекрывает заведующего кафедры)
% \chpretitle{
%     заведующий кафедрой математических основ информатики и олимпиадного\\
%     программирования на базе МАОУ <<Ф"=Т лицей №1>>
% }
% \chtitle{г. Саратов, к.\,ф.-м.\,н., доцент}
% \chname{Кондратова\, Ю.\,Н.}

% Научный руководитель (для реферата преподаватель проверяющий работу)
\satitle{старший преподаватель} %должность, степень, звание
\saname{М.\,И.\,Сафрончик}

% Руководитель практики от организации (руководитель для цифровой кафедры)
\patitle{доцент, к.\,ф.-м.\,н.}
\paname{И.\,А.\,Батраева}

% Руководитель НИР
\nirtitle{доцент, к.\,п.\,н.} % степень, звание
\nirname{И.\,А.\,Батраева}

% Семестр (только для практики, для остальных типов работ не используется)
\term{2}

% Наименование практики (только для практики, для остальных типов работ не
% используется)
\practtype{учебная}

% Продолжительность практики (количество недель) (только для практики, для
% остальных типов работ не используется)
\duration{2}

% Даты начала и окончания практики (только для практики, для остальных типов
% работ не используется)
\practStart{01.07.2024}
\practFinish{13.01.2024}

% Год выполнения отчета
\date{2026}

\maketitle

% Включение нумерации рисунков, формул и таблиц по разделам (по умолчанию -
% нумерация сквозная) (допускается оба вида нумерации)
\secNumbering

\tableofcontents

% Раздел "Обозначения и сокращения". Может отсутствовать в работе
% \abbreviations
% \begin{description}
%     \item ... "--- ...
%     \item ... "--- ...
% \end{description}

% Раздел "Определения". Может отсутствовать в работе
% \definitions

% Раздел "Определения, обозначения и сокращения". Может отсутствовать в работе.
% Если присутствует, то заменяет собой разделы "Обозначения и сокращения" и
% "Определения"
% \defabbr

\intro
В данном отчёте производится сравнение времени работы быстрой сортировки (QuickSort), сортировки слиянием (MergeSort) и пирамидальной сортировки (HeapSort).

\section{Код сортировок}
Мною были реализованы три алгоритма сортировки, основанных на сравнении элементов. Их код приведён ниже.

\subsection{Быстрая сортировка}
\small
\inputminted[breaklines]{cpp}{../quick_sort.cpp}
\normalsize

\subsection{Сортировка слиянием}
\small
\inputminted[breaklines]{cpp}{../merge_sort.cpp}
\normalsize

\subsection{Пирамидальная сортировка}
\small
\inputminted[breaklines]{cpp}{../heap_sort.cpp}
\normalsize

\section{Измерение скорости работы}
Для измерения скорости работы сортировок я генерировала массив случайных чисел размера $n$ и выполняла его сортировку. Для замерения времени я использовала следующий фрагмент кода.
\small
\inputminted[breaklines]{cpp}{./code_time.cpp}
\normalsize

\section{Результаты сравнения}

\begin{figure}[H]
\centering
	\includegraphics[scale=0.5]{results.png}
\end{figure}

% После введения — серии \section, \subsection и т.д.

\conclusion

Наилучшую зависимость времени от размера входных данных показала быстрая сортировка. Тем не менее, в некоторых реализациях быстрой сортировки возможно подобрать худший случай, при котором время её работы составит близкое к $O(n^2)$ значение, а значит, далеко не во всех ситуациях этот алгоритм сортировки наиболее эффективен.

% Отобразить все источники. Даже те, на которые нет ссылок.
% \nocite{*}

\bibliographystyle{ugost2008}
\bibliography{thesis}

% Окончание основного документа и начало приложений Каждая последующая секция
% документа будет являться приложением
\appendix

\end{document}
